\documentclass{article}
\usepackage[utf8]{inputenc}
\usepackage{amsmath}
\begin{document}

\begin{flushleft}
Dear sir,
\end{flushleft} %\newline
I wasn't feeling well for two days. Hence I couldn't make much progress.
I wanted to meet you since I feel that it is not possible to use the final derived equations of this paper.  Following are some of the definitions as in the paper.
\begin{align}
S_{z} &=  \frac{1}{2} \left(\sigma_{x} + \tau_{x} \right)\\
s_{z} &=  \frac{1}{2} \left(\sigma_{x} - \tau_{x} \right) \\
S_{\pm}&= \frac{1}{2} \left( \sigma_{y} \pm i\sigma_{z}  \right) \left( \tau_{y} \pm i\tau_{z} \right)\\
s_{\pm}&= \frac{1}{2} \left( \sigma_{y} \pm i\sigma_{z}  \right) \left( \tau_{y} \mp i\tau_{z} \right)
\end{align}
% $ S_{z} =  \frac{1}{2} \left(\sigma_{x} + \tau_{x} \right)$ and $ %s_{z} =  \frac{1}{2} \left(\sigma_{x} - \tau_{x} \right) $ .   $ S_{+} = %$
%as defined in it
The Hamiltonian for the two qubits is as follows 
 \[ H  = \frac{\omega_{1}}{2} \sigma_{z} \otimes I +\frac{\omega_{2}}{2}I \otimes \sigma_{z} + g(t)\sigma_{x} \otimes \sigma_{x}
 \]
 I fail to see how one can write the above Hamilton in terms of $S_{z}, s_{z}, S_{\pm}, s_{\pm}$
 
I have struggled with this issue at length and yet failed to reach any conclusion.  

(Type your content here.)
%Similarly it also defines
\end{document}
